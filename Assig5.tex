\documentclass[journal,12pt,twocolumn]{IEEEtran}

\usepackage{enumitem}
\usepackage{tfrupee}
\usepackage{amsmath}
\usepackage{amssymb}
\usepackage{gensymb}
\usepackage{graphicx}
\usepackage{txfonts}

\def\inputGnumericTable{}

\usepackage[latin1]{inputenc}
\usepackage{color}
\usepackage{array}
\usepackage{longtable}
\usepackage{calc}
\usepackage{multirow}
\usepackage{hhline}
\usepackage{ifthen}
\usepackage{caption}
\captionsetup[table]{skip=3pt}
\providecommand{\pr}[1]{\ensuremath{\Pr\left(#1\right)}}
\providecommand{\cbrak}[1]{\ensuremath{\left\{#1\right\}}}
\renewcommand{\thefigure}{\arabic{table}}
\renewcommand{\thetable}{\arabic{table}}
\title{Assignment 5 \\ \Large AI1110: Probability and Random Variables \\ \large Indian Institute of Technology Hyderabad}
\author{Rudransh Mishra \\ \normalsize AI21BTECH11025 \\ \vspace*{20pt} \normalsize  29 May 2022 \\ \vspace*{20pt} PROBABILITY, RANDOM VARIABLES, AND STOCHASTIC PROCESSES\\ \normalsize Athanasios Papoulis}

\begin{document}
% The title
\maketitle

% The question
\textbf{Example 6-50}
Suppose that the random variables x and y are $N(0,0,\sigma _1 ^2,\sigma _2 ^2, r)$. As we know,
\begin {align}
  & E\{x^2\}=\sigma _1 ^2\\
  & E\{x^4\}=3 \sigma _1 ^4 
\end {align}
Furthermore, $f (y  | x)$ is a normal density with mean $\frac {r \sigma _2 x}{\sigma _1}$ and variance $\sigma _2 \sqrt {1 - r^2}$. Hence,\\

We shall show that
\begin {align}
  &E\{xy\} = r\sigma _1\sigma _2\\
  &E\{x^2 y^2\} = E\{x^2\}E\{y^2\} + 2E^2\{xy\}
\end {align}

% The answer
\textbf{Proof.}
\begin {align}
&E\{xy\} =E\{xE\{y|x\}\} \\
&=E\{r \sigma_2 \frac{x^ 2} {\sigma_1} \} \\
&=r \sigma_2 \frac{\sigma_1 ^ 2}{\sigma_1}
\end {align}

Now, we know that,

\begin {align}
&E\{x^ 2 y^ 2 \} =E\{x^ 2 E\{y^ 2 |x\} \}\\ 
&=E\{x^ 2 [r^ 2 \sigma_2 ^ 2\frac{ x^ 2} {\sigma_1 ^ 2} + \sigma_2 ^ 2 (1-r^ 2 )]\}\\
&=3 \sigma_1 ^ 4 r^ 2 \frac{\sigma_2 ^ 2} {\sigma_1 ^ 2} + \sigma_1 ^ 2 \sigma_2 ^ 2 (1-r^ 2 )\\
&= \sigma_1 ^ 2 \sigma_2 ^ 2 +2r^ 2 \sigma_1 ^ 2 \sigma_2 ^ 2
\end {align}
and the proof is complete.

\begin{center} ---x-x-x-x-x-x--- \end{center}
\end{document}